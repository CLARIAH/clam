\title{CLAM Documentation}

\begin{document}


\chapter{Introduction} 

The Computational Linguistics Application Mediator (CLAM) allows you to quickly and transparently transform your Natural Language Processing application into a webservice, with which both human end-users as well as automated clients can interact. CLAM takes a description of your system and wraps itself around it, allowing end-users to upload input files to your application, start your application with specific parameters of their choice, and download and view the output of the application. Whilst the application runs, users can monitor its status.

CLAM is set up in a universal fashion, making it flexible enough to be wrapped around a wide range of computational linguistic applications. These applications are treated as a black box, of which only the parameters, input formats and output formats need to be described. The applications themselves need not be network aware in any way, and the handling and validation of input take can be taken care of by CLAM.

The kind of applications that CLAM is intended for are NLP applications, usually of a kind that do some processing on a text corpus. This corpus or smaller text can be uploaded by the user, or may be pre-installed for the webservice. The NLP application is usually expected to produce a certain output, which is subsequently made available through the webservice for viewing and downloading.

The CLAM webservice is a RESTful webservice, meaning it uses the HTTP verbs GET, POST, PUT and DELETE to manipulate resources. The principal resource in CLAM is called a project. Users can maintain various projects, each representing one specific run of the system, with specific input, output and parameters.

The webservice provides its responses in the CLAM XML format, an associated XSL stylesheet can directly transform this to xhtml in the user's browser, thus allowing both machines and people to directly communicate with the webservice.

This documentation is split into two parts: a section for service providers/developers and a section for service clients, users those wanting to communicate with an existing service in an automated fashion.

\chapter{Documentation for Service Providers}

\section{Technical details}

CLAM is written in Python 2.5, and is built on the webpy framework. It can run stand-alone thanks to the built-in cherrypy webserver. So no additional webserver is needed. Note that the software is designed for UNIX systems only, as any serious work in the field of NLP is UNIX based anyhow. 


\subsection{Installation}

The following software is required to run CLAM. These will be available in any modern Linux or BSD distribution:

\begin{itemize}
\item python 2.5 (or a higher 2.x version)
\item python-webpy, version 0.3
\item python-lxml, version 2 or higher
%\item python-libxslt1
%\item python-django
\end{itemize}

Python will run on any TCP port of your choice. It is recommended not to use HTTP port 80. Also make sure the port is open in the firewall, at least for the machines you want to allow access.

To install CLAM, simply uncompress the clam software archive in any desired target location. The following files may be of particular interest to service providers:

\begin{itemize}
\item \texttt{clamservice.py} -- This is the webservice itself, the command to be invoked to start it.
\item \texttt{config/} -- This directory contains service configuration files. Place you service configuration here.
\item \texttt{config/defaults.py} -- This is a default configuration template which you can copy to make your own service configuration.
\item \texttt{common/} -- Common Python modules for CLAM
\item \texttt{common/parameters.py} -- Parameter-type definitions
\item \texttt{common/format.py} -- Format-type definitions
\item \texttt{static/style.css} -- The styling for visualisation

\end{itemize}

Starting the service is done by launching clamservice.py with the name of your service configuration and optionally a port number (the default is 8080).

\texttt{\$ clamservice.py config.ucto 8080}

\section{Architecture}

CLAM has a layered architecture, with at the core the NLP application(s) you want to turn into a webservice. A CLAM webservice needs the following three components from the service developer:

\begin{enumerate}
\item A service configuration file
\item A wrapper script for your NLP application
\item An NLP application
\end{enumerate}

The wrapper script is not stricly mandatory if the NLP application can be directly invoked by CLAM. However, for more complex applications, writing a wrapper script is strongly recommended, as it offers more flexibility and better integration.

%\section{Resource oriented approach}

%Being a RESTful webservice, CLAM has a resource-oriented approach. 


\section{Service configuration}

The service configuration consists out of a description of your NLP application, or rather, the wrapper script that surrounds it. It specifies what parameters the system can take, and what input and output formats are expected. The service configuration is itself a Python script. But due to its straightforwards nature, knowledge of Python is not required to make your own.

The server configuration files reside in the \texttt{config/} directory. Making a new webservice starts with copying the sample \texttt{defaults.py} and editing your copy.

One of the first things to configure is the root path. All projects will be confined to the \texttt{projects/} directory under this root path, each project having its own subdirectory. Pre-installed corpora should be put in the \texttt{corpora/} directory.
 
\subsection{Command}
\label{sec:command}

Central in the configuration file is the command that CLAM will execute. This command should start the actual NLP application, or preferably a script wrapped around it. Full shell syntax is supported and there are some special variables that will be automatically set by CLAM:

\begin{itemize}
\item \texttt{\$INPUTDIRECTORY} - The absolute path to the input directory where all the input files from the user will be stored (possible in subdirectories).
\item \texttt{\$OUTPÛTDIRECTORY} - The absolute path to the output directory. Your system should output all of its files here, as otherwise they are not accessible through CLAM.
\item \texttt{\$STATUSFILE} - The absolute path to a status file. Your system may write a short message to this status file, indicating the current status. This message will be displayed to the user in CLAM's interface. The status file consists only of the latest message.
\text \texttt{\$PARAMETERS} - This variable will contain all parameter flags and the parameter values that have been selected by the user.
\item \texttt{\$DATAFILE} - This is the absolute path to the data file that CLAM outputs in the project directory. This data file, in CLAM XML format, contains all parameters along with their selected values. Furthermore it contains the inputformats and outputformats, and a listing of uploaded input files and/or pre-installed corpora. System wrapper scripts can read this file to obtain all necessary information. If the system wrapper script is written in Python, the CLAM Client API can be used to read this file, required little effort on the part of the developer.
\item \texttt{\$USERNAME} - The username of the logged in user.
\end{itemize}


Make sure the actual command is an absolute path or the script is in the user's \textt{\$PATH}. Upon launch, the current working directory will be automatically set to the specific project directory.


%Within this directory, there will be an input/ and output/ directory, the full path is stored in \texttt{\$INPUTDIRECTORY} respectively \texttt{\$OUTPUTDIRECTORY} and can be passed as arguments to your script. All uploaded user input will be in this input directory, and all output that users should be able to view or download, should be in this output directory.


\subsection{Parameters}

The parameters which an NLP application, or rather the wrapper script, can take, are defined in the service configuration. First of all parameters can be subdivided into parameter groups, but these serve only presentational purposes. 

Parameters of six types are predefined, though other types can be added to \texttt{common/parameters.py}. Each is a Python class taking the following mandatory arguments:

\begin{enumerate}
\item \textbf{\texttt{id}} -- An id for internal use only.
\item \textbf{\texttt{paramflag}} -- The parameter flag, this flag will be added to \texttt{\$PARAMETERS} when the parameter is set. It is customary for parameter flags to consist of a hyphen and a letter or two hyphens and a string. Parameter flags could be for example be formed like: \texttt{-p} ,\texttt{--pages}, \texttt{--pages=}. There will be a space between the parameter flag and its value, unless it ends in a \texttt{=} sign or \texttt{nospace=True} is set. Multi-word string values will automatically be enclosed in quotation marks, for the shell to correctly parse them.
\item \textbf{\texttt{name}} -- The name of this parameter, this will be shown to the user in the interface.
\item \textbf{\texttt{description}} -- A description of this parameter, meant for the end-user.
\end{enumerate}


The six parameter types are:

\begin{itemize}
\item \textbf{\texttt{BooleanParameter}} - A parameter that can only be turned on or off, represented in the interface by a checkbox. If it is turned on, the parameter flag is included in \texttt{\$PARAMETERS}, if it is turned off, it is not. If \texttt{reverse=True} is set, it will do the inverse.
\item \textbf{\texttt{IntegerParameter}} - A parameter expecting an integer number. Use \texttt{minrange=}, and \texttt{maxrange=} to restrict the range.
\item \textbf{\texttt{FloatParameter}} - A parameter expecting a float number. Use \texttt{minrange=}, and \texttt{maxrange=} to restrict the range.
\item \textbf{\texttt{StringParameter}} - %TODO
\item \textbf{\texttt{TextParameter}} - %TODO
\item \textbf{\texttt{ChoiceParameter}} - %TODO

\end{itemize}

All parameters can take the following extra named arguments:

%TODO


\chapter{Documentation for Service Clients}


\end{document}
